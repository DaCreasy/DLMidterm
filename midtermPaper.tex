\documentclass{article}

\usepackage[utf8]{inputenc} % allow utf-8 input
\usepackage[T1]{fontenc}    % use 8-bit T1 fonts
\usepackage{hyperref}       % hyperlinks
\usepackage{url}            % simple URL typesetting
\usepackage{booktabs}       % professional-quality tables
\usepackage{amsfonts}       % blackboard math symbols
\usepackage{nicefrac}       % compact symbols for 1/2, etc.
\usepackage{microtype}      % microtypography

\title{Neural Networks Midterm Project}
\author{Benjamin Klybor, Damian Creasy}

\begin{document}
\maketitle

\section{Introduction}
In this paper we set out to improve on the convolutional CIFAR-10 model provided by Keras\footnote{\url{https://github.com/keras-team/keras/blob/master/examples/cifar10_cnn.py}}. However, instead of testing an improved model on the CIFAR-10 dataset, we set out to test the network on the CIFAR-100 dataset. The CIFAR datasets each contain 60000 color images that are 32 wide by 32 pixels high. The CIFAR-10 set has 6000 examples for each of 10 classes, whereas the CIFAR-100 set has 600 examples for each of 100 classes. Currently, the best model for analyzing the CIFAR-100 dataset achieves an accuracy of about 84\%. While in setting out on this project we do not expect to even approach this accuracy, are surprised to find just how well the basic CIFAR-10 network from Keras works for analyzing the CIFAR-100 data. Starting from a baseline accuracy of about 38\%, we make some adjustments to the model to see how they affect the accuracy.

\section{Related Work}

\section{Approach}

\section{Results}

\section{Conclusion}

\end{document}